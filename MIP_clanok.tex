% Metódy inžinierskej práce

\documentclass[10pt,twoside,slovak,a4paper]{article} %article,coursepaper

\usepackage[slovak]{babel}
%\usepackage[T1]{fontenc}
\usepackage[IL2]{fontenc} % lepšia sadzba písmena Ľ než v T1
\usepackage[utf8]{inputenc}
\usepackage{graphicx}
\usepackage{url} % príkaz \url na formátovanie URL
\usepackage{hyperref} % odkazy v texte budú aktívne (pri niektorých triedach dokumentov spôsobuje posun textu)

\usepackage{cite}
%\usepackage{times}

\pagestyle{headings}

\title{Sebaregulované vzdelávanie\thanks{Semestrálny projekt v predmete Metódy inžinierskej práce, ak. rok 2020/21, vedenie:Ing. Zuzana Špitalová}} % meno a priezvisko vyučujúceho na cvičeniach

\author{Tomáš Nemec\\[2pt]
	{\small Slovenská technická univerzita v Bratislave}\\
	{\small Fakulta informatiky a informačných technológií}\\
	{\small \texttt{xnemect1@stuba.sk}}
	}

\date{\small 25. september 2020} % upravte



\begin{document}

\maketitle

\begin{abstract}
%\ldots
\end{abstract}



\section{Úvod}

%       AHOJKY     VLOZI text
Motivujte čitateľa a vysvetlite, o čom píšete. Úvod sa väčšinou nedelí na časti.

Uveďte explicitne štruktúru článku. Tu je nejaký príklad.
Základný problém,\cite{Kang:FODA} ktorý bol naznačený v úvode, je podrobnejšie vysvetlený v časti~\ref{nejaka}.
Dôležité súvislosti sú uvedené v častiach~\ref{dolezita} a~\ref{dolezitejsia}.
Záverečné poznámky prináša časť~\ref{zaver}.



\section{Sebaregulované vzdelávanie} \label{nejaka}

V článku sa budem venovať téme Sebaregulovaného vzdelávania, kde v úvode
by som chcel podrobnejšie popísať, čo to vlastne sebaregulované vzdelávanie
je, v čom nám môže byť prospešné, aké su jeho výhody a aj nevýhody.




Z obr.~\ref{f:rozhod} je všetko jasné. 

\begin{figure*}[tbh]
\centering
%\includegraphics[scale=1.0]{diagram.pdf}
Aj text môže byť prezentovaný ako obrázok. Stane sa z neho označný plávajúci objekt. Po vytvorení diagramu zrušte znak \texttt{\%} pred príkazom \verb|\includegraphics| označte tento riadok ako komentár (tiež pomocou znaku \texttt{\%}).
\caption{Rozhodujúci argument.}
\label{f:rozhod}
\end{figure*}



\section{Výhody sebaregulovaného vzdelávania} \label{ina}




\subsection{Ešte nejaké vysvetlenie} \label{ina:este}

\paragraph{Veľmi dôležitá poznámka.}
Niekedy je potrebné nadpisom označiť odsek. Text pokračuje hneď za nadpisom.



\section{Nevýhody} \label{dolezita}




\section{Proces} \label{dolezitejsia}
Kedže sa
tento typ vzdelávania skladá z troch častí, tak sa ďalej pokúsim konkrétne
rozobrať, čo všetko každa jedna časť zahŕňa, ako sa v dannej časti postupuje a
na ktoré konkrétne body by sa mal človek pri vykonávaní úlohy zamerať.

\subsection{1.krok}
\subsection{2.krok}
\subsection{3.krok}






\section{Záver} \label{zaver} % prípadne iný variant názvu
Na koniec mam v pláne uviesť nejaké pomôcky, ako napríklad mobilné aplikácie,
pomocou ktorých si vie človek tento štýl vzdelávania zefektívniť a
zautomatizovať.



%\acknowledgement{Ak niekomu chcete poďakovať\ldots}


% týmto sa generuje zoznam literatúry z obsahu súboru literatura.bib podľa toho, na čo sa v článku odkazujete
\bibliography{zdroje}
\bibliographystyle{abbrv} % prípadne alpha, abbrv alebo hociktorý iný
\end{document}
